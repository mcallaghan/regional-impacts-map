\documentclass{article}
\usepackage[a4paper, landscape, total={260mm, 180mm}]{geometry}
\usepackage{authblk}
\usepackage{booktabs}
\usepackage{array}
\usepackage[table]{xcolor}
\usepackage{listings}
\usepackage[T1]{fontenc}

\title{Machine learning-based evidence and attribution mapping of 100,000 climate impact studies - Supplementary material}
\author[1,2]{Max Callaghan}
\author[3,5]{Carl-Friedrich Schleussner}
\author[3,4]{Shruti Nath}
\author[3]{Quentin Lejeune}
\author[6]{Thomas R. Knutson}
\author[7,8]{Markus Reichstein}
\author[]{Gerrit Hansen}
\author[3,5]{Emily Theokritoff}
\author[3,5]{Marina Andrijevic}
\author[3,9]{Robert Brecha}
\author[3]{Michael Hegarty}
\author[3]{Chelsea Jones}
\author[3]{Kaylin Lee}
\author[3]{Agathe Lucas}
\author[3,5]{Nicole van Maanen}
\author[3]{Inga Menke}
\author[3,5]{Peter Pfleiderer}
\author[3]{Burcu Yesil}
\author[1,2]{Jan Minx}




\affil[1]{Mercator Research Institute on Global Commons and Climate Change, Torgauer Straße, 10829 Berlin, Germany}
\affil[2]{Priestley International Centre for Climate, University of Leeds, Leeds LS2 9JT, United Kingdom}
\affil[3]{Climate Analytics, Berlin, Germany}
\affil[4]{Institute of Atmospheric and Climate Sciences, ETH Zürich, Switzerland}
\affil[5]{Integrative Research Institute on Transformations of Human-Environment Systems, Humboldt University, Berlin, Germany}
\affil[6]{NOAA/Geophysical Fluid Dynamics Laboratory. Princeton, NJ, 08540, USA}
\affil[7]{Max Planck Institute for Biogeochemistry, Department Biogeochemical Integration, D-07701 Jena, Germany}
\affil[8]{Michael Stifel Center Jena for Data-driven and Simulation Science, Jena, Germany}
\affil[9]{Hanley Sustainability Institute, Renewable and Clean Energy Program and Physics Dept., University of Dayton, Dayton, Ohio, USA}

\date{}

\newcolumntype{L}[1]{>{\raggedright\let\newline\\\arraybackslash\hspace{0pt}}m{#1}}

\renewcommand{\arraystretch}{1.5}
\begin{document}
	\maketitle

	\section{Search query}
	\begin{lstlisting}[breaklines,breakatwhitespace=true]
	
	
	(TS=("climate model" OR "elevated* temperatur" OR "ocean* warming" OR "saline* intrusion" OR "chang* climat" OR "environment* change" OR "climat* change" OR "climat* warm" OR "warming* climat" OR "climat* varia" OR "global* warming" OR "global* change" OR "greenhouse* effect" OR "snow cover" OR "extreme temperature" OR "cyclone" OR "ocean acidification" OR "anthropogen*" OR "sea* level" OR "precipitation variabil*" OR "precipitation change*" OR "temperature* impact" OR "environmental* variab" OR "weather* pattern" OR "weather* factor*" OR "climat*") OR TS=("change* NEAR/5 cryosphere" OR "increase* NEAR/3 temperatur*")
	) 
	AND 
	(TS=("migration" OR "impact*" OR "specie*" OR "mortality*" OR "health" OR "disease*" OR "ecosystem*" OR "mass balance" OR "flood*" OR "drought" OR "disease*" OR "adaptation" OR "malaria" OR "fire" OR "water scarcity" OR "water supply" OR "permafrost" OR "biological response" OR "food availability" OR "food security" OR "vegetation dynamic*" OR "cyclone*" OR "yield*" OR "gender" OR "indigenous" OR "conflict" OR "inequality" OR "snow water equival*" OR "surface temp*") OR TS=("glacier* NEAR/3 melt*" OR "glacier* NEAR/3 mass*" OR "erosion* NEAR/5 coast*" OR "glacier* NEAR/5 retreat*" OR "rainfall* NEAR/5 reduc*" OR "coral* NEAR/5 stress*" OR "precip* NEAR/5 *crease*" OR "river NEAR/5 flow")
	) 
	AND 
	(TS=("recent" OR "current" OR "modern" OR "observ*" OR "evidence*" OR "past" OR "local" OR "region*" OR "significant" OR "driver*" OR "driving” OR "respon*" OR "were responsible" OR "was responsible" OR "exhibited" OR "witnessed" OR "attribut*" OR "has increased" OR "has decreased" OR "histor*" OR "correlation" OR "evaluation") 
	)
	
	\end{lstlisting}

	\section{Supplementary Data}
	
	\subsection{Category Scheme}
	The file \textbf{category\_aggregation.csv} lists the categories used to code relevant documents. Each category could be used as an impact or a driver. To make the classification problem tractable, the categories were merged into “broad categories” resembling those used in IPCC AR5. 
	
	\subsection{Document metadata}
	The file \textbf{0c\_doc\_info.csv} contains basic metadata about each of the studies considered. For rights reasons the abstracts are not included, although these are used in the analysis.
	
	\subsection{Predictions}
	
	Predicted scores are given in the following files for relevance \textbf{1\_document\_relevance.csv}, impact category \textbf{1\_impact\_predictions.csv}, and climate drivers \textbf{1\_driver\_predictions.csv}.
	
	\subsection{Grid cells}
	
	The file \textbf{2\_merged\_da\_data.csv} shows the attribution category and weighted studies score (for temperature, precipitation and for all studies) for each grid cell.
	
	\subsection{Study grid cell matching}
	The file \textbf{study\_gridcell\_2.5.csv} has a row for each grid cell study combination, where the column `ndf\_id' refers to the `index' column of the aforementioned grid cell dataset.
	
	\subsection{Study D\&A}
	The file \textbf{2\_study\_da.csv} shows the number of gridcells with each detection and attribution category that each study refers to.
	
	\section{Supplementary Tables}	
	\begin{table}
		\rowcolors{2}{gray!10}{white}
		\scriptsize
		\center
		\begin{tabular}{ll |p{1.5cm} p{1.5cm} p{1.5cm} p{1.5cm} p{1.5cm} p{1.5cm} p{1.5cm}}
\toprule
                 &       & Coastal and marine Ecosystems (WS>1) & Human and managed systems (WS>1) & Mountains, snow and ice (WS>1) & Rivers, lakes, and soil moisture (WS>1) & Terrestrial ecosystems (WS>1) &       Other systems (WS>1) &                Total (WS>5) \\
\midrule
South America & D\&A &                 365 \mbox{(194-662)} &             460 \mbox{(275-925)} &           296 \mbox{(184-491)} &                    384 \mbox{(207-794)} &        1366 \mbox{(699-2681)} &      830 \mbox{(331-1744)} &     3674 \mbox{(2400-5760)} \\
                 & Other &                  209 \mbox{(74-520)} &              130 \mbox{(61-322)} &             57 \mbox{(35-116)} &                    234 \mbox{(141-469)} &           235 \mbox{(94-592)} &        205 \mbox{(56-689)} &      1061 \mbox{(605-1995)} \\
North America & D\&A &              2429 \mbox{(1304-4241)} &           1708 \mbox{(936-3300)} &         1734 \mbox{(972-3203)} &                 2621 \mbox{(1415-4899)} &      7835 \mbox{(4308-13552)} &   5614 \mbox{(2485-11662)} &  21745 \mbox{(14364-31884)} \\
                 & Other &               1620 \mbox{(579-3859)} &            446 \mbox{(180-1240)} &          608 \mbox{(295-1259)} &                  1715 \mbox{(876-3595)} &       2259 \mbox{(1029-4821)} &     2307 \mbox{(799-6068)} &    8868 \mbox{(5002-15196)} \\
Africa & D\&A &                 448 \mbox{(219-881)} &           1102 \mbox{(625-2039)} &           268 \mbox{(143-514)} &                   747 \mbox{(393-1422)} &        1556 \mbox{(706-2951)} &     1246 \mbox{(470-2725)} &     5323 \mbox{(3391-8104)} \\
                 & Other &                 345 \mbox{(105-865)} &             364 \mbox{(164-956)} &             79 \mbox{(44-199)} &                   531 \mbox{(259-1077)} &         447 \mbox{(182-1036)} &       496 \mbox{(99-1488)} &     2251 \mbox{(1220-4105)} \\
Europe & D\&A &               1300 \mbox{(628-2426)} &           1637 \mbox{(934-3162)} &          985 \mbox{(547-1937)} &                  1307 \mbox{(743-2638)} &      6006 \mbox{(3389-10396)} &    2850 \mbox{(1150-6705)} &   13991 \mbox{(9105-21466)} \\
                 & Other &                678 \mbox{(222-1900)} &              162 \mbox{(70-620)} &           344 \mbox{(198-723)} &                   834 \mbox{(439-1850)} &         816 \mbox{(361-2020)} &      956 \mbox{(287-2893)} &     3762 \mbox{(2089-7232)} \\
Asia & D\&A &               1164 \mbox{(604-2097)} &          3516 \mbox{(2124-6329)} &         1105 \mbox{(700-2011)} &                 2968 \mbox{(1790-5292)} &      5822 \mbox{(3270-10040)} &   6480 \mbox{(3089-12295)} &  20885 \mbox{(14705-29783)} \\
                 & Other &                712 \mbox{(256-1856)} &            917 \mbox{(410-2383)} &           275 \mbox{(144-531)} &                 1987 \mbox{(1138-3699)} &         991 \mbox{(402-2269)} &     1938 \mbox{(664-4896)} &    6764 \mbox{(3999-11548)} \\
Oceania & D\&A &                890 \mbox{(499-1484)} &             494 \mbox{(271-979)} &           349 \mbox{(191-622)} &                    521 \mbox{(296-993)} &        1539 \mbox{(777-2866)} &     1720 \mbox{(875-3228)} &     5482 \mbox{(3579-8202)} \\
                 & Other &                594 \mbox{(235-1239)} &              117 \mbox{(50-367)} &             79 \mbox{(44-192)} &                    272 \mbox{(125-568)} &          355 \mbox{(155-873)} &      519 \mbox{(215-1336)} &     1922 \mbox{(1047-3441)} \\
Global & D\&A &              5529 \mbox{(2957-9755)} &         7729 \mbox{(4487-14423)} &        3783 \mbox{(2201-7050)} &                6623 \mbox{(3777-12384)} &    19622 \mbox{(10796-34361)} &  14498 \mbox{(6382-30282)} &  57366 \mbox{(38371-85227)} \\
                 & Other &              3465 \mbox{(1277-8215)} &           1870 \mbox{(823-5013)} &         1423 \mbox{(750-2912)} &                 4416 \mbox{(2391-8910)} &       4249 \mbox{(1885-9546)} &   5149 \mbox{(1658-13868)} &  20419 \mbox{(11697-35705)} \\
Without location & D\&A &                       0 \mbox{(0-0)} &                   0 \mbox{(0-0)} &                 0 \mbox{(0-0)} &                          0 \mbox{(0-0)} &                0 \mbox{(0-0)} &             0 \mbox{(0-0)} &              0 \mbox{(0-0)} \\
                 & Other &              3613 \mbox{(1561-7413)} &          2587 \mbox{(1213-6704)} &         1101 \mbox{(575-2266)} &                  1766 \mbox{(887-3867)} &     11117 \mbox{(5839-21759)} &   3956 \mbox{(1517-11551)} &  23954 \mbox{(13897-42921)} \\
\bottomrule
\end{tabular}

		\caption{The number of studies in each impact category and each continent}
	\end{table}

	\begin{table}
		\rowcolors{3}{white}{gray!10}
		\scriptsize
		\center
		\begin{tabular}{ll p{1cm} p{1cm} p{1cm} p{1cm} p{1cm} p{1cm} p{1cm} p{1cm} p{1cm} p{1cm} p{1cm} p{1cm} p{1cm} p{1cm}}
\toprule
       &       & \multicolumn{2}{L{2cm}}{Coastal and marine Ecosystems (WS>1)} & \multicolumn{2}{L{2cm}}{Human and managed systems (WS>1)} & \multicolumn{2}{L{2cm}}{Mountains, snow and ice (WS>1)} & \multicolumn{2}{L{2cm}}{Rivers, lakes, and soil moisture (WS>1)} & \multicolumn{2}{L{2cm}}{Terrestrial ecosystems (WS>1)} & \multicolumn{2}{L{2cm}}{Other systems (WS>1)} & \multicolumn{2}{L{2cm}}{Total (WS>5)} \\
       &       &                                 area &               population &                             area &               population &                           area &               population &                                    area &               population &                          area &               population &                     area &               population &                     area &               population \\
\midrule
South America & D\&A &                8\% \mbox{(5\%-31\%)} &  26\% \mbox{(18\%-51\%)} &          26\% \mbox{(13\%-62\%)} &  55\% \mbox{(38\%-79\%)} &         11\% \mbox{(8\%-15\%)} &  19\% \mbox{(14\%-32\%)} &                  16\% \mbox{(6\%-45\%)} &  33\% \mbox{(14\%-68\%)} &       63\% \mbox{(38\%-64\%)} &  77\% \mbox{(67\%-81\%)} &  45\% \mbox{(11\%-64\%)} &  63\% \mbox{(27\%-81\%)} &  52\% \mbox{(18\%-63\%)} &  75\% \mbox{(46\%-81\%)} \\
       & Other &                3\% \mbox{(1\%-27\%)} &    6\% \mbox{(2\%-15\%)} &           24\% \mbox{(5\%-34\%)} &   14\% \mbox{(9\%-17\%)} &         13\% \mbox{(6\%-14\%)} &    6\% \mbox{(3\%-10\%)} &                  19\% \mbox{(3\%-34\%)} &   15\% \mbox{(5\%-19\%)} &       36\% \mbox{(29\%-36\%)} &  19\% \mbox{(17\%-19\%)} &   28\% \mbox{(8\%-36\%)} &   17\% \mbox{(7\%-19\%)} &  33\% \mbox{(12\%-36\%)} &  19\% \mbox{(13\%-19\%)} \\
North America & D\&A &              33\% \mbox{(19\%-50\%)} &  58\% \mbox{(43\%-70\%)} &          40\% \mbox{(25\%-57\%)} &  64\% \mbox{(48\%-72\%)} &        34\% \mbox{(18\%-54\%)} &  43\% \mbox{(27\%-56\%)} &                 52\% \mbox{(24\%-64\%)} &  60\% \mbox{(41\%-69\%)} &       70\% \mbox{(61\%-70\%)} &  70\% \mbox{(64\%-72\%)} &  58\% \mbox{(32\%-71\%)} &  72\% \mbox{(58\%-72\%)} &  62\% \mbox{(55\%-70\%)} &  70\% \mbox{(69\%-72\%)} \\
       & Other &               11\% \mbox{(5\%-12\%)} &   27\% \mbox{(9\%-27\%)} &            9\% \mbox{(7\%-15\%)} &  28\% \mbox{(26\%-28\%)} &          8\% \mbox{(5\%-19\%)} &   19\% \mbox{(8\%-24\%)} &                  15\% \mbox{(9\%-18\%)} &  28\% \mbox{(28\%-28\%)} &       21\% \mbox{(16\%-21\%)} &  28\% \mbox{(28\%-28\%)} &  16\% \mbox{(11\%-21\%)} &  28\% \mbox{(27\%-28\%)} &  17\% \mbox{(16\%-21\%)} &  28\% \mbox{(28\%-28\%)} \\
Africa & D\&A &               12\% \mbox{(4\%-23\%)} &  22\% \mbox{(12\%-37\%)} &          32\% \mbox{(25\%-46\%)} &  68\% \mbox{(58\%-79\%)} &           3\% \mbox{(1\%-9\%)} &    7\% \mbox{(6\%-14\%)} &                  17\% \mbox{(5\%-35\%)} &  44\% \mbox{(18\%-68\%)} &       40\% \mbox{(22\%-51\%)} &  71\% \mbox{(46\%-79\%)} &  35\% \mbox{(15\%-52\%)} &  64\% \mbox{(26\%-79\%)} &  32\% \mbox{(24\%-41\%)} &  66\% \mbox{(51\%-75\%)} \\
       & Other &                 2\% \mbox{(0\%-4\%)} &     1\% \mbox{(0\%-4\%)} &            7\% \mbox{(3\%-14\%)} &     3\% \mbox{(3\%-5\%)} &           0\% \mbox{(0\%-1\%)} &     0\% \mbox{(0\%-1\%)} &                   9\% \mbox{(1\%-13\%)} &     4\% \mbox{(2\%-7\%)} &         7\% \mbox{(3\%-20\%)} &     4\% \mbox{(2\%-9\%)} &    9\% \mbox{(2\%-23\%)} &     6\% \mbox{(2\%-8\%)} &    6\% \mbox{(2\%-12\%)} &     4\% \mbox{(2\%-6\%)} \\
Europe & D\&A &              16\% \mbox{(11\%-21\%)} &  59\% \mbox{(41\%-71\%)} &          25\% \mbox{(20\%-33\%)} &  80\% \mbox{(73\%-88\%)} &         21\% \mbox{(8\%-30\%)} &  58\% \mbox{(25\%-79\%)} &                 24\% \mbox{(18\%-31\%)} &  77\% \mbox{(64\%-86\%)} &       35\% \mbox{(28\%-56\%)} &  88\% \mbox{(83\%-95\%)} &  29\% \mbox{(21\%-45\%)} &  85\% \mbox{(73\%-93\%)} &  28\% \mbox{(25\%-34\%)} &  83\% \mbox{(78\%-90\%)} \\
       & Other &                 0\% \mbox{(0\%-0\%)} &     0\% \mbox{(0\%-0\%)} &             1\% \mbox{(0\%-1\%)} &     0\% \mbox{(0\%-0\%)} &           0\% \mbox{(0\%-1\%)} &     0\% \mbox{(0\%-0\%)} &                    0\% \mbox{(0\%-0\%)} &     0\% \mbox{(0\%-0\%)} &          1\% \mbox{(1\%-3\%)} &     0\% \mbox{(0\%-0\%)} &     0\% \mbox{(0\%-1\%)} &     0\% \mbox{(0\%-0\%)} &     0\% \mbox{(0\%-1\%)} &     0\% \mbox{(0\%-0\%)} \\
Asia & D\&A &               19\% \mbox{(6\%-29\%)} &  47\% \mbox{(18\%-58\%)} &          61\% \mbox{(44\%-78\%)} &  80\% \mbox{(75\%-82\%)} &        27\% \mbox{(14\%-43\%)} &  39\% \mbox{(12\%-56\%)} &                 61\% \mbox{(34\%-72\%)} &  75\% \mbox{(50\%-81\%)} &       71\% \mbox{(60\%-81\%)} &  80\% \mbox{(72\%-83\%)} &  73\% \mbox{(51\%-83\%)} &  80\% \mbox{(73\%-84\%)} &  70\% \mbox{(60\%-76\%)} &  80\% \mbox{(78\%-82\%)} \\
       & Other &                 3\% \mbox{(1\%-4\%)} &    9\% \mbox{(4\%-13\%)} &             5\% \mbox{(5\%-7\%)} &  15\% \mbox{(15\%-15\%)} &           4\% \mbox{(2\%-4\%)} &   14\% \mbox{(6\%-14\%)} &                    5\% \mbox{(5\%-5\%)} &  15\% \mbox{(15\%-15\%)} &          6\% \mbox{(4\%-7\%)} &  15\% \mbox{(14\%-15\%)} &     6\% \mbox{(5\%-7\%)} &  15\% \mbox{(15\%-15\%)} &     5\% \mbox{(5\%-6\%)} &  15\% \mbox{(15\%-15\%)} \\
Oceania & D\&A &              76\% \mbox{(72\%-76\%)} &  91\% \mbox{(59\%-91\%)} &          74\% \mbox{(53\%-84\%)} &  70\% \mbox{(54\%-86\%)} &           3\% \mbox{(2\%-8\%)} &  37\% \mbox{(15\%-53\%)} &                  19\% \mbox{(4\%-83\%)} &  53\% \mbox{(43\%-82\%)} &       84\% \mbox{(79\%-86\%)} &  77\% \mbox{(74\%-92\%)} &  86\% \mbox{(73\%-86\%)} &  89\% \mbox{(61\%-91\%)} &  84\% \mbox{(74\%-86\%)} &  84\% \mbox{(75\%-92\%)} \\
       & Other &              14\% \mbox{(14\%-14\%)} &     5\% \mbox{(5\%-5\%)} &          14\% \mbox{(14\%-14\%)} &     5\% \mbox{(5\%-5\%)} &           1\% \mbox{(1\%-3\%)} &     4\% \mbox{(4\%-4\%)} &                  14\% \mbox{(1\%-14\%)} &     5\% \mbox{(4\%-5\%)} &       14\% \mbox{(14\%-14\%)} &     5\% \mbox{(5\%-5\%)} &  14\% \mbox{(14\%-14\%)} &     5\% \mbox{(5\%-5\%)} &  14\% \mbox{(14\%-14\%)} &     5\% \mbox{(5\%-5\%)} \\
Global & D\&A &              22\% \mbox{(13\%-33\%)} &  44\% \mbox{(23\%-57\%)} &          38\% \mbox{(27\%-53\%)} &  73\% \mbox{(64\%-79\%)} &         17\% \mbox{(9\%-27\%)} &  32\% \mbox{(13\%-46\%)} &                 32\% \mbox{(16\%-49\%)} &  62\% \mbox{(40\%-75\%)} &       53\% \mbox{(41\%-62\%)} &  75\% \mbox{(63\%-81\%)} &  49\% \mbox{(29\%-63\%)} &  75\% \mbox{(59\%-82\%)} &  48\% \mbox{(37\%-56\%)} &  74\% \mbox{(67\%-80\%)} \\
       & Other &                 5\% \mbox{(2\%-8\%)} &    8\% \mbox{(3\%-11\%)} &            8\% \mbox{(4\%-12\%)} &  11\% \mbox{(11\%-12\%)} &           4\% \mbox{(2\%-7\%)} &    9\% \mbox{(4\%-10\%)} &                   9\% \mbox{(3\%-12\%)} &  11\% \mbox{(10\%-12\%)} &        12\% \mbox{(9\%-15\%)} &  12\% \mbox{(11\%-13\%)} &   11\% \mbox{(5\%-16\%)} &  12\% \mbox{(11\%-13\%)} &   10\% \mbox{(6\%-13\%)} &  12\% \mbox{(11\%-12\%)} \\
\bottomrule
\end{tabular}

		\caption{The percentage of land area and population with robust evidence by impact category and continent}
	\end{table}

	\begin{table}
		\rowcolors{3}{white}{gray!10}
		\scriptsize
		\center
		\begin{tabular}{ll p{1cm} p{1cm} p{1cm} p{1cm} p{1cm} p{1cm} p{1cm} p{1cm} p{1cm} p{1cm} p{1cm} p{1cm} p{1cm} p{1cm}}
\toprule
       &       & \multicolumn{2}{L{2cm}}{Coastal and marine Ecosystems (WS>1)} & \multicolumn{2}{L{2cm}}{Human and managed systems (WS>1)} & \multicolumn{2}{L{2cm}}{Mountains, snow and ice (WS>1)} & \multicolumn{2}{L{2cm}}{Rivers, lakes, and soil moisture (WS>1)} & \multicolumn{2}{L{2cm}}{Terrestrial ecosystems (WS>1)} & \multicolumn{2}{L{2cm}}{Other systems (WS>1)} & \multicolumn{2}{L{2cm}}{Total (WS>5)} \\
       &       &                                 area &               population &                             area &               population &                           area &               population &                                    area &               population &                          area &               population &                     area &               population &                     area &               population \\
\midrule
Low income & D\&A &                8\% \mbox{(2\%-20\%)} &   17\% \mbox{(5\%-27\%)} &          35\% \mbox{(29\%-45\%)} &  53\% \mbox{(46\%-61\%)} &          6\% \mbox{(3\%-11\%)} &    6\% \mbox{(3\%-17\%)} &                  22\% \mbox{(7\%-37\%)} &  36\% \mbox{(11\%-54\%)} &       37\% \mbox{(25\%-48\%)} &  51\% \mbox{(41\%-61\%)} &  38\% \mbox{(17\%-54\%)} &  51\% \mbox{(27\%-64\%)} &  34\% \mbox{(27\%-43\%)} &  50\% \mbox{(43\%-59\%)} \\
       & Other &                 3\% \mbox{(0\%-5\%)} &  12\% \mbox{(11\%-17\%)} &            5\% \mbox{(4\%-11\%)} &  20\% \mbox{(20\%-21\%)} &           1\% \mbox{(1\%-2\%)} &   16\% \mbox{(5\%-17\%)} &                    4\% \mbox{(2\%-8\%)} &  20\% \mbox{(18\%-23\%)} &         7\% \mbox{(3\%-19\%)} &  21\% \mbox{(18\%-26\%)} &    8\% \mbox{(4\%-20\%)} &  22\% \mbox{(19\%-24\%)} &     5\% \mbox{(3\%-7\%)} &  20\% \mbox{(19\%-22\%)} \\
Lower middle income & D\&A &               26\% \mbox{(9\%-37\%)} &  59\% \mbox{(19\%-65\%)} &          50\% \mbox{(42\%-74\%)} &  77\% \mbox{(71\%-80\%)} &         22\% \mbox{(6\%-29\%)} &  49\% \mbox{(12\%-58\%)} &                 38\% \mbox{(19\%-59\%)} &  67\% \mbox{(37\%-77\%)} &       64\% \mbox{(41\%-71\%)} &  78\% \mbox{(58\%-80\%)} &  58\% \mbox{(31\%-72\%)} &  76\% \mbox{(62\%-80\%)} &  58\% \mbox{(40\%-65\%)} &  77\% \mbox{(70\%-79\%)} \\
       & Other &                 5\% \mbox{(1\%-8\%)} &   10\% \mbox{(2\%-16\%)} &           14\% \mbox{(9\%-19\%)} &  18\% \mbox{(18\%-18\%)} &         12\% \mbox{(3\%-12\%)} &   17\% \mbox{(7\%-17\%)} &                  13\% \mbox{(8\%-20\%)} &  18\% \mbox{(17\%-19\%)} &       19\% \mbox{(14\%-21\%)} &  19\% \mbox{(17\%-19\%)} &   15\% \mbox{(7\%-22\%)} &  18\% \mbox{(17\%-19\%)} &   16\% \mbox{(9\%-21\%)} &  18\% \mbox{(18\%-19\%)} \\
Upper middle income & D\&A &               11\% \mbox{(5\%-23\%)} &  29\% \mbox{(17\%-49\%)} &          34\% \mbox{(21\%-51\%)} &  80\% \mbox{(69\%-88\%)} &         13\% \mbox{(9\%-24\%)} &  19\% \mbox{(11\%-42\%)} &                 31\% \mbox{(16\%-47\%)} &  73\% \mbox{(52\%-85\%)} &       53\% \mbox{(39\%-64\%)} &  87\% \mbox{(81\%-91\%)} &  45\% \mbox{(25\%-59\%)} &  83\% \mbox{(67\%-90\%)} &  45\% \mbox{(33\%-53\%)} &  85\% \mbox{(77\%-89\%)} \\
       & Other &                 2\% \mbox{(1\%-9\%)} &     4\% \mbox{(1\%-6\%)} &            9\% \mbox{(3\%-11\%)} &     6\% \mbox{(4\%-6\%)} &           2\% \mbox{(2\%-3\%)} &     2\% \mbox{(0\%-4\%)} &                   9\% \mbox{(2\%-12\%)} &     6\% \mbox{(4\%-6\%)} &        10\% \mbox{(8\%-13\%)} &     6\% \mbox{(5\%-6\%)} &   10\% \mbox{(4\%-13\%)} &     6\% \mbox{(5\%-6\%)} &   10\% \mbox{(5\%-12\%)} &     6\% \mbox{(5\%-6\%)} \\
High Income & D\&A &              45\% \mbox{(34\%-56\%)} &  77\% \mbox{(60\%-85\%)} &          50\% \mbox{(37\%-66\%)} &  85\% \mbox{(80\%-90\%)} &        32\% \mbox{(16\%-48\%)} &  68\% \mbox{(37\%-82\%)} &                 46\% \mbox{(24\%-68\%)} &  81\% \mbox{(71\%-86\%)} &       73\% \mbox{(66\%-76\%)} &  87\% \mbox{(86\%-89\%)} &  67\% \mbox{(47\%-78\%)} &  89\% \mbox{(81\%-91\%)} &  67\% \mbox{(60\%-74\%)} &  88\% \mbox{(85\%-89\%)} \\
       & Other &                9\% \mbox{(6\%-10\%)} &     9\% \mbox{(6\%-9\%)} &            8\% \mbox{(6\%-13\%)} &     9\% \mbox{(8\%-9\%)} &          5\% \mbox{(3\%-13\%)} &     7\% \mbox{(6\%-8\%)} &                  12\% \mbox{(5\%-14\%)} &     9\% \mbox{(8\%-9\%)} &       15\% \mbox{(12\%-17\%)} &     9\% \mbox{(9\%-9\%)} &   13\% \mbox{(9\%-17\%)} &     9\% \mbox{(9\%-9\%)} &  13\% \mbox{(12\%-16\%)} &     9\% \mbox{(9\%-9\%)} \\
Global & D\&A &              22\% \mbox{(13\%-33\%)} &  44\% \mbox{(23\%-57\%)} &          38\% \mbox{(27\%-53\%)} &  73\% \mbox{(64\%-79\%)} &         17\% \mbox{(9\%-27\%)} &  32\% \mbox{(13\%-46\%)} &                 32\% \mbox{(16\%-49\%)} &  62\% \mbox{(40\%-75\%)} &       53\% \mbox{(41\%-62\%)} &  75\% \mbox{(63\%-81\%)} &  49\% \mbox{(29\%-63\%)} &  75\% \mbox{(59\%-82\%)} &  48\% \mbox{(37\%-56\%)} &  74\% \mbox{(67\%-80\%)} \\
       & Other &                 5\% \mbox{(2\%-8\%)} &    8\% \mbox{(3\%-11\%)} &            8\% \mbox{(4\%-12\%)} &  11\% \mbox{(11\%-12\%)} &           4\% \mbox{(2\%-7\%)} &    9\% \mbox{(4\%-10\%)} &                   9\% \mbox{(3\%-12\%)} &  11\% \mbox{(10\%-12\%)} &        12\% \mbox{(9\%-15\%)} &  12\% \mbox{(11\%-13\%)} &   11\% \mbox{(5\%-16\%)} &  12\% \mbox{(11\%-13\%)} &   10\% \mbox{(6\%-13\%)} &  12\% \mbox{(11\%-12\%)} \\
\bottomrule
\end{tabular}

		\caption{The percentage of land area and population with robust evidence by impact category and income category}
	\end{table}

	\begin{table}
		\rowcolors{3}{white}{gray!10}
		\scriptsize
		\center
		\begin{tabular}{ll p{1cm} p{1cm} p{1cm} p{1cm} p{1cm} p{1cm} p{1cm} p{1cm} p{1cm} p{1cm} p{1cm} p{1cm} p{1cm} p{1cm}}
\toprule
       &       & \multicolumn{2}{L{2cm}}{Gridcell sums} \\
       &       &          area & population \\
\midrule
South America & D\&A &          64\% &       81\% \\
       & Other &          36\% &       19\% \\
North America & D\&A &          71\% &       72\% \\
       & Other &          29\% &       28\% \\
Africa & D\&A &          73\% &       89\% \\
       & Other &          27\% &       11\% \\
Europe & D\&A &          95\% &      100\% \\
       & Other &           5\% &        0\% \\
Asia & D\&A &          93\% &       85\% \\
       & Other &           7\% &       15\% \\
Oceania & D\&A &          86\% &       95\% \\
       & Other &          14\% &        5\% \\
Global & D\&A &          80\% &       87\% \\
       & Other &          20\% &       13\% \\
\bottomrule
\end{tabular}

		\caption{The percentage of land area and population with attributable impacts in temperature and/or precipitation by continent}
	\end{table}
	
	\begin{table}
		\rowcolors{3}{white}{gray!10}
		\scriptsize
		\center
		\begin{tabular}{ll p{1cm} p{1cm} p{1cm} p{1cm} p{1cm} p{1cm} p{1cm} p{1cm} p{1cm} p{1cm} p{1cm} p{1cm} p{1cm} p{1cm}}
\toprule
       &       & \multicolumn{2}{L{2cm}}{Gridcell sums} \\
       &       &          area & population \\
\midrule
5. Low income & D\&A &          74\% &       73\% \\
       & Other &          26\% &       27\% \\
4. Lower middle income & D\&A &          78\% &       81\% \\
       & Other &          22\% &       19\% \\
3. Upper middle income & D\&A &          85\% &       94\% \\
       & Other &          15\% &        6\% \\
1. High Income & D\&A &          78\% &       91\% \\
       & Other &          22\% &        9\% \\
Global & D\&A &          80\% &       87\% \\
       & Other &          20\% &       13\% \\
\bottomrule
\end{tabular}

		\caption{The percentage of land area and population with attributable impacts in temperature and/or precipitation by income category}
	\end{table}
	
\end{document}